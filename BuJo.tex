\documentclass[a5paper]{article}
\usepackage{latexsym}
\usepackage[english]{babel} % To obtain English text with the blindtext package
\usepackage[normalem]{ulem} %Strikethrough text
\title{Bullet Journal}
\begin{document}
\maketitle
\tableofcontents
\newpage
\section{Future log}
\begin{table}[htp]
	\begin{tabular}
		\hline
		Jan
		\begin{itemize}
			\item lorem ipsum 
			\item lorem ipsum 
			\item Proper Node
			\item Proper Node
		\end{itemize}
		\hline
		Feb
		\begin{itemize}
			\item lorem ipsum 
			\item lorem ipsum 
			\item Proper Node
			\item Proper Node
		\end{itemize}
		\hline
		Mar
		\begin{itemize}
			\item lorem ipsum 
			\item lorem ipsum 
			\item Proper Node
			\item Proper Node
		\end{itemize}
		\hline
	\end{tabular}
\end{table}
\newpage
\begin{table}[htp]
	\begin{tabular}
		\hline
		Jan
		\begin{itemize}
			\item lorem ipsum 
			\item lorem ipsum 
			\item Proper Node
			\item Proper Node
		\end{itemize}
		\hline
		Feb
		\begin{itemize}
			\item lorem ipsum 
			\item lorem ipsum 
			\item Proper Node
			\item Proper Node
		\end{itemize}
		\hline
		Mar
		\begin{itemize}
			\item lorem ipsum 
			\item lorem ipsum 
			\item Proper Node
			\item Proper Node
		\end{itemize}
		\hline
	\end{tabular}
\end{table}
\newpage
\section{Month view}
\begin{itemize}
	\item Items are numbered automatically.
	\item The numbers start at 1 with each use of the \texttt{enumerate} environment.
	\item Another entry in the list
	\item Another entry in the list
	\item Another entry in the list
	\item Another entry in the list
	\item Another entry in the list
	\item Another entry in the list
	\item Another entry in the list
	\item Another entry in the list
	\item Another entry in the list
	\item Another entry in the list
	\item Another entry in the list
	\item Another entry in the list
	\item Another entry in the list
	\item Another entry in the list
	\item Another entry in the list
	\item Another entry in the list
	\item Another entry in the list
	\item Another entry in the list
	\item Another entry in the list
	\item Another entry in the list
	\item Another entry in the list
	\item Another entry in the list
	\item Another entry in the list
	\item Another entry in the list
	\item Another entry in the list
	\item Another entry in the list
	\item Another entry in the list
	\item Another entry in the list
	\item Another entry in the list
\end{itemize}	
\newpage
\section{Task lists}
Legend:
\begin{itemize}
	\item Task incomplete
	\item[x] Task complete 
	\item[$<$] Task migrated into collection
	\item[$>$] Task scheluded into future log
	\item \sout{Task irrelevant} %Strikethrough text
	\item[-] Notes
	\item[*] Important/Priority
	\item[!] Inspiration
\end{itemize}
\newpage
\section{Daily view}
09.12.2021
\begin{itemize}
	\item idea to create bullet journal.tex
\end{itemize}
10.12.2021
\begin{itemize}
	\item first prototype bujo.tex
\end{itemize}
20.12.2021
\begin{itemize}
	\item published on GitHub!
\end{itemize}
26.01.2022
\begin{itemize}
	\item PDF conversion
\end{itemize}
\newpage
\section{Collection}
\textbf{Rapid Logging} - Be it for taking notes or journaling, studies keep identifying benefits of writing by hand. That said, it takes time and can be unorganized. How can we enjoy the benefits while avoiding the shortcomings of hand writing? Rapid Logging. Rapid Logging is the language in which the Bullet Journal is written. In short, it's a way of capturing information as bulleted lists. Let's start with the basics.

\textbf{Bullets} - If Rapid Logging is language the BuJo is written in, Bullets are the syntax. Bullets are short-form sentences paired with symbols that visually categorize your entries into: Tasks, Events, or Notes. Let's break it down...
\begin{itemize}
	\item textbf{Tasks:} things you have to do
	\item[-] textbf{Notes} things you don't want to forget 
	\item[O] textbf{Events}: Noteworth moments in time 
\end{itemize}
\textbf{Tasks} - Tasks are represented by a simple dot ... of a checkbox because it's fast, clean, and can easily be transformed to reflect the state of the Task. Tasks can have one of five states:
\begin{enumerate}
	\item Task incomplete.
	\item[x] Task complete.
	\item[$<$] Task migrated into collection
	\item[$>$] Task scheluded into future log
	\item \sout{Task irrelevant} %Strikethrough text
\end{enumerate}

\textbf{Events} - Events are represented by the open circle "O" Bullet. Events are date-related entries that can either be scheduled (e.g. "Charlie's birthday") or logged after they occur (e.g. "signed the lease"). Our experiences can be complicated and distracting. Rather than trying to capture the way you feel in the moment, keep your Event entries short and objective. It will increase the odds of us writing them down. The important thing is to have a record of your experience so that you can learn from it

\begin{description}
	\item Tom's last day
	\item Watched Wonder woman in UK
	\item[O] Walked home to clear head
\end{description}

\textbf{Notes} - Notes are represented with a dash -. Notes include: facts, ideas, thoughts, and observations. They're used to capture information or data you don't want to forget. This Bullet works well for meeting, lecture, or classroom notes.

\begin{itemize}
	\item[-] Mar 25: Office closed for reno.
	\item[-] Jack has a gluten intolerancje
	\item[-] Use mayo instead of butter on grilled cheese
\end{itemize}

\textbf{Mix \& Match} - Tasks, Events, and Notes will help you quickly capture your thoughts as they bubble up throughout the day. Don't worry about logging them in any particular order. The important thing is to get them out of your head, and onto the page.

Change the labels using \verb|\item[label text]| in an \texttt{enumerate} environment
	\begin{itemize}
		\item[O] Leigh dinner
		\item Claire: organize birthday cake 
		\item Niclas: call re travel docs 
		\item[-] Song about vending machine!
		\item[O] Watched wonder woman in UK
	\end{itemize}
	\textbf{Nesting} - Nesting Bullets can add some much needed color to your entries. For example, nest Notes under an Event to capture important details. Nest subtasks under the main Task to break things down into a series of steps. 
	\begin{itemize}
		\item[O] Leigh dinner 
			\begin{itemize}
				\item[-] elevators 
				\item[-] Talked about how to help Jane 
			\end{itemize}
		\item Claire: organized birthday cake
			\begin{itemize}
				\item Order cake
				\item Pick up cake
				\item[-] Needs to be gluten free
			\end{itemize}
	\end{itemize}

	\textbf{Signifiers} - Signifiers are symbols that give your entries additional context at a glance. They're placed to the left of Bullets so they stick out, making them easy to spot when scanning your pages. Here are two useful examples, but feel free to come up with your own.
	\begin{itemize}
		\item[*] = Priority: Used to mark the most important things on your list. Use it sparingly. If everything is a priority, nothing is.
		\item[!] = Inspiration: Great ideas, personal mantras, and genius insights will never be misplaced again!
	\end{itemize}
	\begin{itemize}
		\item[O]Claire: organize birthday cake
			\begin{itemize}
				\item[*] Order cake
				\item Pick up cake
				\item[-] Needs to be gluten free
			\end{itemize}
		\item Claire: organized birthday cake
			\begin{itemize}
				\item Order cake
				\item Pick up cake
				\item[-] Needs to be gluten free
			\end{itemize}
		\item Niclas: call re travel docs 
		\item[!] Song about vending machine!
	\end{itemize}
\end{document}
